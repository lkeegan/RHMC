\documentclass[a4paper,12pt]{article}

%PACKAGES
\usepackage[margin=2cm]{geometry}
\usepackage{latexsym,amsmath,amsfonts,amssymb}
\usepackage{graphicx,color}
\usepackage{pstricks,pst-node,pst-text,pst-3d}
\usepackage{verbatim}
\usepackage{hyperref}
\usepackage{float,placeins}
\usepackage{xcolor}
\hypersetup{
    colorlinks,
    linkcolor={reds!50!black},
    citecolor={blue!50!black},
    urlcolor={blue!50!black}
}
%USEFUL COMMANDS
\newcommand{\tr}[1]{\text{Tr}\big[#1\big]}
\newcommand{\ga}[1]{\Gamma\left(#1\right)}
\renewcommand{\d}[1]{\delta^d\Big(#1\Big)}
\newcommand{\ie}{\emph{i.e.}}
\newcommand{\ex}[1]{{\rm e}^{#1}} \def\ii{{\rm i}}
\newcommand{\Tr}{{\rm Tr}}
\newcommand{\sect}[1]{\setcounter{equation}{0}\section{#1}}
\renewcommand{\theequation}{\thesection.\arabic{equation}}

\begin{document}

\author{Liam Keegan\\ \href{mailto:liam@keegan.ch}{liam@keegan.ch}}
\title{$\mu$RHMC Code Documentation \\ \href{https://github.com/lkeegan/muRHMC}{github.com/lkeegan/muRHMC}}

\maketitle

\section{Introduction}
A simple RHMC code to simulate $n_f+n_f$ QCD with isospin chemical potential $\mu_u = -\mu_d \equiv \mu_I$, using unimproved staggered fermions and the Wilson SU(3) gauge action.

\section{QCD formulation}
\subsection{Gauge Action}
Wilson SU(3) plaquette lattice gauge action,
\begin{equation}
 S_g[U] = -\frac{\beta}{3} \sum_x \sum_{\mu<\nu} Re \Tr \left[ U_{\mu}(x)  U_{\nu}(x+\hat\mu)  U_{\mu}^{\dagger}(x+\hat\nu)  U_{\nu}^{\dagger}(x)\right]
\end{equation}
where $U_\mu(x)$ is a 3x3 complex matrix at the site $x$ in the direction $\mu$.

Defining the staple
\begin{equation}
 A_\mu(x) \equiv \sum_{\mu\neq\nu} \left\{
 U_{\nu}(x+\hat\mu) U_{\mu}^{\dagger}(x+\hat\nu)  U_{\nu}^{\dagger}(x) +
 U_{\nu}^{\dagger}(x+\hat\mu-\hat\nu) U_{\mu}^{\dagger}(x-\hat\nu)  U_{\nu}(x-\hat\nu) \right\}
\end{equation}
the terms in the action that depend on a given link $U_{\mu}(x)$ are then
\begin{equation}
 S_g^{local}[U_{\mu}(x)] = -\frac{\beta}{6} \Tr \left[ U_{\mu}(x) A_{\mu}(x) + A_{\mu}^{\dagger}(x)U_{\mu}^{\dagger}(x)  \right]
\end{equation}
which is related to the full action by
\begin{equation}
 S_g[U] = \frac{1}{12} \sum_{x} \sum_{\mu} S_g^{local}[U_{\mu}(x)]
\end{equation}

\subsection{Fermion Action}
The fermion action for $4+4$ staggered flavors with isospin chemical potential $\mu_u = -\mu_d = \mu_I$ and mass $m$ is given by
\begin{equation}
\label{eq:action_fermion}
 S_f[U] = \phi^{\dagger} [D(\mu_I,m) D^{\dagger}(\mu_I,m)]^{-1} \phi
\end{equation}
where $D(\mu_I,m)$ is the staggered lattice Dirac operator
\begin{align}
 [D(\mu_I,m)]_{xy} &= e^{\mu_I/2} U_{0}(x) \delta_{n,m-\hat0} - e^{-\mu_I/2} U_{0}^{\dagger}(x-\hat0) \delta_{n,m+\hat0} \\
 &+ \sum_{\mu=1}^{3} \eta_{\mu}(x) \left[ U_{\mu}(x) \delta_{n,m-\hat\mu} - U_{\mu}^{\dagger}(x-\hat\mu) \delta_{n,m+\hat\mu} \right]
\end{align}
and $\eta_{\mu}(x) = -1^{\sum_{\nu<\mu}x_{\nu}}$ are the space-dependent staggered equivalent of Dirac $\gamma$-matrices, i.e.
\begin{align}
 \eta_{0}(x) &= 1 \\
 \eta_{1}(x) &= -1^{x_0} \\
 \eta_{2}(x) &= -1^{x_0+x_1} \\
 \eta_{3}(x) &= -1^{x_0+x_1+x_2} \\
 \eta_{5}(x) &= -1^{x_0+x_1+x_2+x_3}
\end{align}
For $\mu_I=0$ this reduces to the standard $n_f=8$ staggered action, with the usual staggered $\gamma_5$--hermicity property $D^{\dagger} = \eta_5 D \eta_5 = -D$ for the massless case, 
which allows us to write 
\begin{equation}
D(0,m) D^{\dagger}(0,m) = (D(0,0) + m)(D(0,0) + m)^{\dagger} = D(0,0)D^{\dagger}(0,0) + m^2
\end{equation}
Moreover, if we split the pseudofermion field into even and odd sites we find a block diagonal operator
\begin{equation}
D(0,m) D(0,m)^{\dagger} \phi = 
\begin{pmatrix}
D(0,0)D^{\dagger}(0,0) + m^2 & 0 \\
0 & D(0,0)D^{\dagger}(0,0) + m^2
\end{pmatrix}
\begin{pmatrix}
\phi^{even} \\
\phi^{odd}
\end{pmatrix}
\end{equation}
so we can find the determinant of just one of these blocks and square it, halving the size of the effective Dirac operator. Equivalently an LU preconditioning of the Dirac operator 
applied to even and odd sites allows us to compute the inverse of the operator by inverting a hermitian operator applied to only even sites.

However, for non--zero $\mu_I$ the $\gamma_5$--hermicity property becomes instead $D(\mu_I)^{\dagger} = \eta_5 D(-\mu_I) \eta_5 = -D(-\mu_I)$. This means that the above matrix is 
no longer block diagonal, but instead 
\begin{equation}
D(\mu_I,m) [D(\mu_I,m)]^{\dagger} \phi = 
\begin{pmatrix}
D(\mu_I,0)D^{\dagger}(\mu_I,0) + m^2 & m\hat D(\mu_I) \\
 m\hat D(\mu_I) & D(\mu_I,0)D^{\dagger}(\mu_I,0) + m^2
\end{pmatrix}
\begin{pmatrix}
\phi^{even} \\
\phi^{odd}
\end{pmatrix}
\end{equation}
where
\begin{equation}
 [\hat D(\mu_I)]_{xy} = \sinh(\mu_I/2)\left[U_0(x)\delta_{x,y-\hat0} + U_0(x-\hat0)\delta_{x,y+\hat0}\right]
\end{equation}
and the LU preconditioning results in a non-hermitian operator (which is ok but means BiCGstab instead of CG will need to be used for the inversion).

\section{HMC}
The hamiltonian for the HMC is
\begin{equation}
 S[P, U, \phi] = S_p[P] + S_g[U] + S_f[\phi, U]
\end{equation}
where we need to solve the classical equations of motion
\begin{align}
 \frac{dP_{\mu}(x)}{dt} &= -\frac{\partial}{\partial U_{\mu}(x)}\left(S_g[U] + S_f[\phi, U]\right) = -F_g(x,\mu) - F_f(x,\mu) \\
 \frac{dU_{\mu}(x)}{dt} &= P_{\mu}(x)
\end{align}
The following steps are required to update a gauge configuration $U$:
\begin{enumerate}
 \item Generate gaussian momenta $P$
 \item Generate $\chi$ with distribution $e^{-\chi^{\dagger}\chi}$, then set $\phi = D\chi$
 \item Integrate the force terms to generate $U', P'$
 \item Accept or reject $U'$ with probability $e^{S[P, U, \phi] - S[P', U', \phi]}$
\end{enumerate}

\subsection{Momenta}
The momenta $P_{\mu}(x)$ are defined as
\begin{equation}
 P_{\mu}(x) = \sum_{a=1}^8 p^a_{\mu}(x) T_a 
\end{equation}
where $T_a$ are the generators of SU(3) and $p^a_{\mu}(x)$ are real numbers.
The action is given by
\begin{equation}
 S_p[P] = \sum_{x,\mu}\Tr\left\{\left[P_{\mu}(x)\right]^2\right\} = \frac{1}{2}\sum_{x,\mu}\sum_{a=1}^8 [p^a_{\mu}(x)]^2
\end{equation}
so we can generate momenta simply by sampling the numbers $p^a$ from the normal gaussian distribution $e^{-(p^a)^2/2}$.

The resulting matrices $P$ have $\Tr[P]=0$ and $\left\langle \Tr [P^2]\right\rangle = 4$.

\subsection{Pseudofermions}
The pseudofermion fields are 3-component complex vectors defined on each site of the lattice. To generate a field $\phi$ according to the distribution $e^{-S_f[U]}$ we 
can first generate a complex vector $\chi$ with distribution $e^{-\chi^{\dagger}\chi}$, i.e. set each of 
the six real numbers that make up the 3 complex components of $\chi$ to a random value $r$ sampled from the normal gaussian distribution $e^{-r^2/2}$.

Then set $\phi = D(\mu_I,m)\chi$.

\subsection{Gauge Force}
The gauge force is given by
\begin{equation}
F_g(x,\mu) = \sum F_g^a(x,\mu) T_a
\end{equation}
where
\begin{align}
F_g^a(x,\mu) &= \frac{\partial S_g[U]}{\partial U_{\mu}^{(a)}(x)} = \frac{\partial S_g^{local}[U_{\mu}(x)]}{\partial U_{\mu}^{(a)}(x)} \\ 
&= -\frac{\beta}{6} \frac{\partial }{\partial U_{\mu}^{(a)}(x)} \Tr \left[ U_{\mu}(x) A_{\mu}(x) + A_{\mu}^{\dagger}(x)U_{\mu}^{\dagger}(x)  \right] \\
&= \frac{i\beta}{6} \Tr \left[ T_a \left( U_{\mu}(x) A_{\mu}(x) - A_{\mu}^{\dagger}(x)U_{\mu}^{\dagger}(x)\right) \right]
\end{align}
But the quantity $(...)$ is traceless and anti-hermitian, i.e. can be written as $\sum_b c_b T_b$, so that using the trace identity $Tr[T_aT_b] = \tfrac{1}{2}\delta_{ab}$ we find
\begin{align}
F_g(x,\mu) &= \sum_{ab} \frac{i\beta}{6} \Tr \left[ T_a c_b T_b \right] T_a = \frac{i\beta}{12} \sum_a c_a T_a \\
&= \frac{i\beta}{12} \left[ U_{\mu}(x) A_{\mu}(x) - A_{\mu}^{\dagger}(x)U_{\mu}^{\dagger}(x) \right]
\end{align}

\subsection{Fermion Force}

\subsection{Force Integration}
The integration of the force terms is done by alternating two discrete steps,
\begin{align}
  I_P(P, F, \epsilon): P & \leftarrow P - \epsilon F \\
  I_U(U, P, \epsilon): U & \leftarrow e^{i \epsilon P} U
\end{align}
which can be combined to form reversible integrators of different orders.
\subsection{Leapfrog}
The simplest integrator is the leapfrog, where an integration step of size $\epsilon$ is given by
\begin{itemize}
 \item $I_P(\epsilon/2)$
 \item $I_U(\epsilon)$
 \item $I_P(\epsilon/2)$
\end{itemize}
giving a second order integrator with one Dirac operator inversion required per step.

\subsection{OMF2}
One integration step of size $\epsilon$ is given by
\begin{itemize}
 \item $I_P(\lambda \epsilon)$
 \item $I_U(\epsilon/2)$
 \item $I_P((1-2\lambda) \epsilon)$
 \item $I_U(\epsilon/2)$
 \item $I_P(\lambda \epsilon)$
\end{itemize}
where $\lambda$ is a tunable parameter, e.g. $\lambda = 0.193$. This is also a second order integrator, 
with $\sim10\times$ smaller errors than leapfrog, but requires two Dirac operator inversions per step.

\subsection{OMF4}
todo?

\section{Inverters}
\subsection{CG}
\subsection{BiCGstab}
\subsection{CG-Multishift}
\subsection{BiCGstab-Multishift}
\subsection{BiCGstab-Block}
\subsection{BiCGstab-Block-Multishift}

\section{Observables}
\subsection{Gauge Observables}
\subsubsection{Plaquette}
The plaquette is averaged over the whole lattice and normalised to 1,
\begin{equation}
 plaquette[U] = \frac{1}{L^4} \sum_x \frac{1}{6}\sum_{\mu<\nu} \frac{1}{3} Re \Tr \left[ U_{\mu}(x)  U_{\nu}(x+\hat\mu)  U_{\mu}^{\dagger}(x+\hat\nu)  U_{\nu}^{\dagger}(x)\right]
\end{equation}

\subsubsection{Polyakov Loop}
The polyakov loop is the product of links in the time direction, averaged over the spatial voume, normalised to 1,
\begin{equation}
 polyakov[U] = \frac{1}{L^3} \sum_x^{L^3} \frac{1}{3} Re \Tr \left[\prod_{t=0}^{T-1} U_{0}(x, t)\right]
\end{equation}

\subsection{Fermionic Observables}
Fermionic observables generally involve traces of the inverse of the Dirac operator, which can be estimated stochastically by 
inverting gaussian noise vectors. For this purpose we define here $\phi$ as a gaussian noise vector with unit norm, $\phi^{\dagger}.\phi=1$ NB should this be 1 or 3??,
which we use as a source vector to find
\begin{equation}
 \chi = \left[D(\mu_I,m) D^{\dagger}(\mu_I,m)\right]^{-1} \phi = -\left[D(\mu_I,m) D(-\mu_I,-m)\right]^{-1} \phi
\end{equation}
and also
\begin{equation}
 \psi = -\left[D(\mu_I,m)\right]^{-1} \phi = D(-\mu_I,-m) \chi
\end{equation}
from which we can construct estimators of various fermionic observables.

\subsubsection{Quark condensate}
\begin{equation}
 condensate = -\left\langle \phi^{\dagger}.\psi \right\rangle
\end{equation}

\subsubsection{Pion susceptibility}
\begin{equation}
 susceptibility = \left\langle \phi^{\dagger}.\chi \right\rangle
\end{equation}

\subsubsection{Isospin density}
\begin{equation}
 density = 2 \Re \left\langle \chi^{\dagger}.\left(\frac{\partial D(\mu_I,m)}{\partial\mu_I}\psi\right) \right\rangle
\end{equation}
where
\begin{equation}
 \left[\frac{\partial D(\mu_I,m)}{\partial\mu_I}\psi\right]_x =\frac{\mu_I}{4}\left[ e^{\mu_I/2} U_0(x) \psi(x+\hat0) + e^{-\mu_I/2} U^{\dagger}_0(x-\hat0) \psi(x-\hat0)\right]
\end{equation}

\newpage
\appendix
\section{SU(3) Matrix Algebra}

\subsection{Generators}
The generators of SU(3) are the set of traceless $3\times 3$ hermitian matrices $T_a$, where $a=1,2,\ldots,8$, with the properties
\begin{equation}
\tr{T_a}=0, \quad T_a^{\dagger}=T_a, \quad \tr{T_a T_b} = \tfrac{1}{2}\delta_{ab}
\end{equation}
\begin{equation}
[T_a,T_b] = T_a T_b - T_b T_a = f_{abc} T_c
\end{equation}
where $f_{abc}$ is real and antisymmetric in all indices. An SU(3) matrix $U$ can be written as
\begin{equation}
U = e^{i \omega_a T_a}
\end{equation}
where $\omega_a$ are real numbers.

\subsection{Differentiation}
Differentiation w.r.t an element of the algebra can be defined as
\begin{equation}
\frac{\partial F(U)}{\partial U^{(a)}} \equiv \frac{\partial}{\partial \omega} \left. F(e^{i \omega T_a} U) \right|_{\omega=0}
\end{equation}
which gives for SU(3) matrices $U, V, W,$
\begin{align}
\frac{\partial}{\partial U^{(a)}} \left( V U W \right) &= i V T_a U W \\
\frac{\partial}{\partial U^{(a)}} \left( V U^{\dagger} W \right) &= -i V U T_a W
\end{align}

\end{document}