\documentclass[a4paper,12pt]{article}

%PACKAGES
\usepackage[margin=2cm]{geometry}
\usepackage{latexsym,amsmath,amsfonts,amssymb}
%\usepackage{multirow}
%\usepackage{float,placeins}
%\usepackage[dvips]{graphicx}
%\usepackage[pdftex]{graphicx,color}
\usepackage{graphicx,color}
%\usepackage{feynmf}
%\usepackage[FIGBOTCAP]{subfigure}
%\usepackage{mathrsfs,mathtools}
\usepackage{pstricks,pst-node,pst-text,pst-3d}
\usepackage{verbatim}
%\usepackage{epstopdf}
%\usepackage{epsfig}
\usepackage{hyperref}
\usepackage{float,placeins}

%\usepackage{showkeys}
%\input proof

%USEFUL COMMANDS
\newcommand{\tr}[1]{\text{Tr}\big[#1\big]}
\newcommand{\ga}[1]{\Gamma\left(#1\right)}
\renewcommand{\d}[1]{\delta^d\Big(#1\Big)}
\newcommand{\ie}{\emph{i.e.}}
\newcommand{\ex}[1]{{\rm e}^{#1}} \def\ii{{\rm i}}
\newcommand{\Tr}{{\rm Tr}}
\newcommand{\sect}[1]{\setcounter{equation}{0}\section{#1}}
\renewcommand{\theequation}{\thesection.\arabic{equation}}

% Pagination
% \renewcommand{\baselinestretch}{1.1}
% \textwidth 160mm
% \textheight 215mm
 \topmargin -3cm
% \oddsidemargin 5mm

\begin{document}

\author{Liam Keegan}
\title{RHMC Code Documentation}

\maketitle

\section{Introduction}
A simple RHMC code to simulate $n_f+n_f$ QCD with isospin chemical potential, using unimproved staggered fermions and the Wilson SU(3) gauge action.

\section{QCD formulation}
\subsection{Gauge Action}
Wilson SU(3) plaquette lattice gauge action,
\begin{equation}
 S_g[U] = -\frac{\beta}{3} \sum_x \sum_{\mu<\nu} Re \Tr \left[ U_{\mu}(x)  U_{\nu}(x+\hat\mu)  U_{\mu}^{\dagger}(x+\hat\nu)  U_{\nu}^{\dagger}(x)\right]
\end{equation}
where $U_\mu(x)$ is a 3x3 complex matrix at the site $x$ in the direction $\mu$.

Defining the staple
\begin{equation}
 A_\mu(x) \equiv \sum_{\mu\neq\nu} \left\{
 U_{\nu}(x+\hat\mu) U_{\mu}^{\dagger}(x+\hat\nu)  U_{\nu}^{\dagger}(x) +
 U_{\nu}^{\dagger}(x+\hat\mu-\hat\nu) U_{\mu}^{\dagger}(x-\hat\nu)  U_{\nu}(x-\hat\nu) \right\}
\end{equation}
the terms in the action that depend on a given link $U_{\mu}(x)$ are then
\begin{equation}
 S_g^{local}[U_{\mu}(x)] = -\frac{\beta}{6} \Tr \left[ U_{\mu}(x) A_{\mu}(x) + A_{\mu}^{\dagger}(x)U_{\mu}^{\dagger}(x)  \right]
\end{equation}
which is related to the full action by
\begin{equation}
 S_g[U] = \frac{1}{12} \sum_{x} \sum_{\mu} S_g^{local}[U_{\mu}(x)]
\end{equation}

\subsection{Fermion Action}
The fermion action for $4+4$ staggered flavors with isospin chemical potential $\mu_I$ and mass $m$ is given by
\begin{equation}
\label{eq:action_fermion}
 S_f[U] = \phi^{\dagger} (D(\mu_I,m) [D(\mu_I,m)]^{\dagger})^{-1} \phi
\end{equation}
where $D(\mu_I,m)$ is the staggered lattice Dirac operator
\begin{align}
 [D(\mu_I,m)]_{xy} &= e^{\mu_I/2} U_{0}(x) \delta_{n,m-\hat0} - e^{-\mu_I/2} U_{0}^{\dagger}(x-\hat0) \delta_{n,m+\hat0} \\
 &+ \sum_{\mu=1}^{3} \eta_{\mu}(x) \left[ U_{\mu}(x) \delta_{n,m-\hat\mu} - U_{\mu}^{\dagger}(x-\hat\mu) \delta_{n,m+\hat\mu} \right]
\end{align}
and $\eta_{\mu}(x) = -1^{\sum_{\nu<\mu}x_{\nu}}$ are the space-dependent staggered equivalent of Dirac $\gamma$-matrices, i.e.
\begin{align}
 \eta_{0}(x) &= 1 \\
 \eta_{1}(x) &= -1^{x_0} \\
 \eta_{2}(x) &= -1^{x_0+x_1} \\
 \eta_{3}(x) &= -1^{x_0+x_1+x_2} \\
 \eta_{5}(x) &= -1^{x_0+x_1+x_2+x_3}
\end{align}
For $\mu_I=0$ this reduces to the standard $n_f=8$ staggered action, with the usual staggered $\gamma_5$--hermicity property $D^{\dagger} = \eta_5 D \eta_5 = -D$ for the massless case, 
which allows us to write the operator in the action in the form $DD^{\dagger} + m^2$.

However for non--zero $\mu_I$ we have instead the relation
\begin{equation}
 [D(\mu_I,m)]^{\dagger} = - D(-\mu_I,-m)
\end{equation}
so that
\begin{equation}
 D(\mu_I,m) [D(\mu_I,m)]^{\dagger} = -D(\mu_I,m) D(-\mu_I,-m)
\end{equation}

To generate a field $\phi$ according to the distribution $e^{-S_f[U]}$ we 
can first generate a complex vector $\chi$ with distribution $e^{-\chi^{\dagger}\chi}$, then set $\phi = D(\mu_I,m)\chi$
\section{HMC}
The hamiltonian for the HMC is
\begin{equation}
 S[P, U, \phi] = S_p[P] + S_g[U] + S_f[\phi, U]
\end{equation}
where we need to solve the classical equations of motion
\begin{align}
 \frac{dP_{\mu}(x)}{dt} &= -\frac{\partial}{\partial U_{\mu}(x)}(S_g[U] + S_f[\phi, U]) = -F_g(x,\mu) - F_f(x,\mu) \\
 \frac{dU_{\mu}(x)}{dt} &= P_{\mu}(x)
\end{align}
The steps required to update a gauge configuration $U$:
\begin{itemize}
 \item Generate gaussian momenta $P$
 \item Generate $\chi$ with distribution $e^{-\chi^{\dagger}\chi}$, then set $\phi = D\chi$
 \item Integrate the force terms to generate $U', P'$
 \item Accept or reject $U'$ with probability $e^{S[P, U, \phi] - S[P', U', \phi]}$
\end{itemize}

\subsection{Momenta}
The momenta $P_{\mu}(x)$ are defined as
\begin{equation}
 P_{\mu}(x) = \sum_{a=1}^8 p^a_{\mu}(x) T_a 
\end{equation}
where $T_a$ are the generators of SU(3) and $p^a_{\mu}(x)$ are real numbers.
The action is given by
\begin{equation}
 S_p[P] = \sum_{x,\mu}\Tr\left\{\left[P_{\mu}(x)\right]^2\right\} = \frac{1}{2}\sum_{x,\mu}\sum_{a=1}^8 [p^a_{\mu}(x)]^2
\end{equation}
so we can generate momenta simply by sampling the numbers $p^a$ from the gaussian distribution $e^{-(p^a)^2/2}$.

The resulting matrices $P$ have $\Tr[P]=0$ and $\left\langle \Tr [P^2]\right\rangle = 4$.

\subsection{Gauge Force}
The gauge force is given by
\begin{equation}
F_g(x,\mu) = \sum F_g^a(x,\mu) T_a
\end{equation}
where
\begin{align}
F_g^a(x,\mu) &= \frac{\partial S_g[U]}{\partial U_{\mu}^{(a)}(x)} = \frac{\partial S_g^{local}[U_{\mu}(x)]}{\partial U_{\mu}^{(a)}(x)} \\ 
&= -\frac{\beta}{6} \frac{\partial }{\partial U_{\mu}^{(a)}(x)} \Tr \left[ U_{\mu}(x) A_{\mu}(x) + A_{\mu}^{\dagger}(x)U_{\mu}^{\dagger}(x)  \right] \\
&= \frac{i\beta}{6} \Tr \left[ T_a \left( U_{\mu}(x) A_{\mu}(x) - A_{\mu}^{\dagger}(x)U_{\mu}^{\dagger}(x)\right) \right]
\end{align}
But the quantity $(...)$ is traceless and anti-hermitian, i.e. can be written as $\sum_b c_b T_b$, so that using the trace identity $Tr[T_aT_b] = \tfrac{1}{2}\delta_{ab}$ we find
\begin{align}
F_g(x,\mu) &= \sum_{ab} \frac{i\beta}{6} \Tr \left[ T_a c_b T_b \right] T_a = \frac{i\beta}{12} \sum_a c_a T_a \\
&= \frac{i\beta}{12} \left[ U_{\mu}(x) A_{\mu}(x) - A_{\mu}^{\dagger}(x)U_{\mu}^{\dagger}(x) \right]
\end{align}

\subsection{Fermion Force}

\subsection{Force Integration}
The integration of the force terms is done by alternating two discrete steps,
\begin{align}
 I_P(P, F, \epsilon)&: P \Leftarrow P - \epsilon F \\
 I_U(U, P, \epsilon)&: U \Leftarrow e^{i \epsilon P) U
\end{align}
which can be combined in various ways to give integrators that are exactly reversible.
\subsection{Leapfrog}
The simplest integrator is the leapfrog, where an integration step of size $\epsilon$ is given by
\begin{itemize}
 \item $I_P(\epsilon/2)$
 \item $I_U(\epsilon)$
 \item $I_P(\epsilon/2)$
\end{itemize}

\subsection{OMF2}

\section{Inverters}
\subsection{CG}
\subsection{CG-Multishift}
\subsection{CG-Block}

\appendix
\section{SU(3) Matrix Algebra}

\subsection{Generators}
The generators of SU(3) are the set of traceless $3\times 3$ hermitian matrices $T_a$, where $a=1,2,\ldots,8$, with the properties
\begin{equation}
\tr{T_a}=0, \quad T_a^{\dagger}=T_a, \quad \tr{T_a T_b} = \tfrac{1}{2}\delta_{ab}
\end{equation}
\begin{equation}
[T_a,T_b] = T_a T_b - T_b T_a = f_{abc} T_c
\end{equation}
where $f_{abc}$ is real and antisymmetric in all indices. An SU(3) matrix $U$ can be written as
\begin{equation}
U = e^{i \omega_a T_a}
\end{equation}
where $\omega_a$ are real numbers.

\subsection{Differentiation}
Differentiation w.r.t an element of the algebra can be defined as
\begin{equation}
\frac{\partial F(U)}{\partial U^{(a)}} \equiv \frac{\partial}{\partial \omega} \left. F(e^{i \omega T_a} U) \right|_{\omega=0}
\end{equation}
which gives for SU(3) matrices $U, V, W,$
\begin{align}
\frac{\partial}{\partial U^{(a)}} \left( V U W \right) &= i V T_a U W \\
\frac{\partial}{\partial U^{(a)}} \left( V U^{\dagger} W \right) &= -i V U T_a W
\end{align}

\end{document}